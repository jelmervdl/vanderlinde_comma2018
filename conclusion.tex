In this paper we presented HASL, our implementation of a human language for structured argumentation. It allows us to express arguments with cooperative and independent supporting relations between claims, with all three types of attacks, and with general and specific claims in text. These texts can then be interpreted and visualized in argument diagrams, providing two different views on the same argument. The text itself may be ambiguous, but such is intentional as human language is ambiguous as well. In such cases all possible interpretations are presented, which helps understand the nature of the ambiguity, giving the writer either the information to remove it, or to make the educated decision to allow it.

\paragraph{Future research}

The interpretation and own semantics of the rule-like nature of generalized claims is first explored in this version of HASL, but is limited in scope. This structure is currently used to fill in the missing conclusion or the missing minor premises in arguments in the form of syllogisms. In the future we want to expand on these semantics, and allow another level of understanding in the resolution of enthymemes.

Lastly, the grammar itself is very closely related to the resulting argument structure. It might be possible to formulate a process in which the argument diagram itself is deconstructed in smaller arguments that fit sentences, and through the use of the grammar rules formulated as a textual argument. Such a process would need to make up for the information that is lost when translating a textual argument into an argument diagram, but like with the enthymeme resolution educated guesses can be tried out and ambiguity can be dealt with by presenting a multitude of formulations, preferring the ones that, when interpreted again, yield the least ambiguous interpretation.